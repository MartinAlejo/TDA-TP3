\section{Conclusiones}

En este trabajo práctico, vimos dos versiones del problema de la batalla naval. Por un lado, tenemos un problema NP-Completo, donde intentamos colocar los barcos de forma tal que se cumplan las demandas de todas las filas y columnas, siempre respetando las restricciones de adyacencias. Como demostramos en las secciones anteriores, el problema es de tipo NP y lo pudimos demostrar realizando una reducción polinomial de otro problema NP-Completo como es 3-Partition.

Luego vimos otra variante del problema, en la cual minimizamos la demanda incumplida utilizando un algoritmo de Backtracking.

Por último, implementamos el algoritmo que nos propone John Jellicoe, el cual no nos lleva a la solución óptima, pero nos permite aproximarnos con una cota inferior de $0,3604$, lo cual es aproximádamente un $33\%$ de la solución óptima.