\section{Algoritmo de aproximación}


\subsection{Descripción del algoritmo}

A continuación se presenta un algoritmo de aproximación para el problema de la batalla naval.

\lstinputlisting[language=python]{code/aproximacion.py}

\subsection{Análisis del algoritmo}

El algoritmo de aproximación para el problema de la batalla naval tiene una complejidad de $O((n + m) \cdot b \cdot k)$, donde $n$ es el número de filas, $m$ es el número de columnas del tablero, $b$ es la longitud del barco y $k$ es la cantidad de barcos que hay. Ya que el algoritmo recorre los barcos, compara cual tiene más demanda, si la columna o la fila, que luego define si poner el barco horizontal o vertical. 

\begin{table}[h!]
    \centering
    \caption{Resultados de demandas}
    \label{tab:resultados}
    \begin{tabular}{|c|c|c|c|}
      \hline
      Demanda Total & Demanda Aproximada - A(I) & Demanda Óptima - Z(I) & Relación \( r(A) \) \\ \hline
      11            & 2                         & 4                     & 0.5000             \\ \hline
      18            & 12                        & 12                    & 1.0000             \\ \hline
      53            & 18                        & 26                    & 0.6923             \\ \hline
      40            & 20                        & 40                    & 0.5000             \\ \hline
      58            & 18                        & 46                    & 0.3913             \\ \hline
      67            & 18                        & 40                    & 0.4500             \\ \hline
      120           & 38                        & 104                   & 0.3653             \\ \hline
      247           & 62                        & 172                   & 0.3604             \\ \hline
      360           & 88                        & 202                   & 0.4356             \\ \hline
    \end{tabular}
\end{table}

Como podemos ver, no es el mejor algoritmo para acercarse a la solución óptima, que sería maximizar la demanda cumplida. Sin embargo, en algunos casos llega como por ejemplo en el segundo ejemplo ilustrado anteriormente, o se acerca bastante como el tercer ejemplo. 

La cota a la que llegamos empiricamente, para ver la relación entre el resultado óptimo, es $r(A) = 0.3604$.

Por lo tanto, en el peor de los casos, obtuvimos una aproximación de $3/10$ a la solución óptima.

