\section{Algoritmo de aproximación}


\subsection{Descripción del algoritmo}

A continuacion se presenta un algoritmo de aproximación para el problema de la batalla naval.

\lstinputlisting[language=python]{code/aproximacion.py}

\subsection{Análisis del algoritmo}

El algoritmo de aproximación para el problema de la batalla naval tiene una complejidad de $O((n + m) \cdot b \cdot k)$, donde $n$ es el número de filas, $m$ es el número de columnas del tablero, $b$ es la longitud del barco y $k$ es la cantidad de barcos que hay. Ya que el algoritmo recorre los barcos, compara cual tiene más demanda, si la columna o la fila, que luego define si poner el barco horizontal o vertical. 

La aproximación que nos da el algoritmo es el siguiente:

demanda total:  11

Demanda cumplida:  2

demanda optima: 4

----------------------------

demanda total:  18

Demanda cumplida:  12

demanda optima: 12

----------------------------

demanda total:  53

Demanda cumplida:  18

demanda optima: 26

----------------------------

demanda total:  14

Demanda cumplida:  0

demanda optima: 6

----------------------------

demanda total:  40

Demanda cumplida:  20

demanda optima: 40

----------------------------

demanda total:  58

Demanda cumplida:  18

demanda optima: 46

----------------------------

demanda total:  67

Demanda cumplida:  18

demanda optima: 40

----------------------------

demanda total:  120

Demanda cumplida:  38

demanda optima: 104

----------------------------

demanda total:  247

Demanda cumplida:  62

demanda optima: 172

----------------------------

demanda total:  360

Demanda cumplida:  88

demanda optima: 202

Como podemos ver, no es el mejor algoritmo para acercarse a la solución óptima, que sería maximizar la demanda cumplida. Sin embargo, en algunos casos llega como por ejemplo en el segundo ejemplo ilustrado anteriormente
