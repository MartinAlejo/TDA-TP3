\section{Demostración: NP}

Para que un problema se encuentre en NP, se debe poder encontrar un validador que valide si la solución es correcta, y lo haga en tiempo polinomial.

\vskip0.25cm

Nuestro problema, está dado por:

\begin{itemize}
    \item Una lista con las demandas para cada fila
    \item Una lista con las demandas para cada columna
    \item Una lista de k barcos (donde el barco s tiene $b_{s}$ de largo)
\end{itemize}

La solución, está dada por una Matriz de tamaño n * m, donde para cada casillero ij:

\begin{itemize}
    \item Si no hay barco, entonces Matriz[i][j] = None
    \item Si hay un barco, entonces Matriz[i][j] = s
\end{itemize}

\textbf{Nota}: Siendo s el índice de dicho barco
\vskip0.25cm
El validador entonces verifica lo siguiente:

\begin{itemize}
    \item Que los barcos no sean adyacentes
    \item Que las demandas de las filas se cumplan de manera exacta
    \item Que las demandas de las columnas se cumplan de manera exacta
    \item Que se coloquen todos los barcos
\end{itemize}

\subsection{Que los barcos no sean adyacentes}

    Para verificar que los barcos no sean adyacentes, basta con recorrer cada casillero de la matriz (n * m) y en cada celda visitar las 8 celdas vecinas (un cuadrado).
    En cada una de las celdas vecinas que visito, veo si hay otro barco distinto, y si lo hay, la solución no es válida.
    Visitar las 8 celdas vecinas se hace en tiempo constante O(1). Por lo tanto, verificar que no haya barcos adyacentes tiene un costo de O(n * m)

\subsection{Que las demandas de las filas y columnas se cumplan de manera exacta}

    Para verificar que las demandas de las filas se cumplan, basta con recorrer cada casillero de la matriz (n * m) y en cada casillero que haya un barco, restar 1 a la demanda de la fila y columna a la que pertenece.
    Finalmente, se recorre la lista con las demandas para cada fila y nos fijamos que sea todo igual a cero O(n). Lo mismo con la lista de las demandas para cada columna O(m). 
    Esto nos da un costo total de O(n * m)

\subsection{Que se coloquen todos los barcos}

    Para verificar que se colocaron todos los barcos, hay que recorrer cada casillero de la matriz (n * m), y en el caso de que en el casillero haya un barco, se agrega a los barcos visitados.
    Luego se recorren todos los barcos O(k) y nos fijamos que hayan sido todos colocados. Esto nos da un costo total de O(n * m).

\vskip0.25cm
En conclusion, como pudimos validar una solucion al problema en tiempo polinomial, podemos afirmar que el problema se encuentra en NP.