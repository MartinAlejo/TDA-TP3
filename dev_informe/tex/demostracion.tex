\section{Demostración: NP-Completo}

\subsection*{Reducción desde el problema 3-Partition}

Vamos a demostrar que la batalla naval es $NP$-Completo. Pero, ¿Cómo se demuestra que un problema es $NP$-Completo? Tiene que cumplir dos condiciones: 

\begin{itemize}
    \item \textbf{Pertenencia a NP:} La verificación de una solución candidata es posible en tiempo polinomial.
    \item \textbf{NP-dificultad:} Se puede realizar una reducción polinomial desde cualquier problema NP-completo hacia este problema.
\end{itemize}

Ya en la sección anterior pudimos verificar con éxito que nuestro problema pertenece a $NP$, ahora hay que demostrar que se puede realizar una reducción polinomial desde cualquier problema $NP$-Completo. Recordemos que todos los $NP$-completos pueden ser reducidos a cualquier otro problema $NP$-Completo. 

En nuestro caso, utilizaremos el problema de \textit{3-Partition}, que como vamos a demostrar más abajo, es un problema $NP$-Completo, para verificar que el problema de la batalla naval pertenece a $NP$-Completo. Es decir, se puede realizar la reducción polinomial: \textbf{3-Partition} $\leq_p$ \textbf{PBN}.

\subsubsection*{Definición del problema 3-Partition}

Dado un conjunto de \(3m\) números positivos \(A = \{a_1, a_2, \dots, a_{3m}\}\), donde cada número está representado en notación unaria y la suma total de los números es \(S = m \cdot B\), queremos particionar \(A\) en \(m\) subconjuntos disjuntos \(S_1, S_2, \dots, S_m\) tales que:

\begin{itemize}
    \item Cada subconjunto \(S_i\) tiene exactamente 3 elementos.
    \item La suma de los elementos en cada subconjunto es exactamente \(B\).
\end{itemize}

----------------------------------------------------------------------------------------------------------

\subsection*{Reducción de 3-Partition a La Batalla Naval}

Dado una instancia del problema \textit{3-Partition}, construiremos una instancia del problema \textit{La Batalla Naval}.

\paragraph{Construcción del tablero:}

\begin{itemize}
    \item \textbf{Dimensiones del tablero:}
    \begin{itemize}
        \item Cantidad de filas: \(m \times 2 - 1\). Cada subconjunto \(S_i\) ocupa una fila, y se agrega una fila vacía después de cada subconjunto excepto después del último.
        \item Cantidad de columnas: \(B + 2\). Esto incluye la suma de cada subconjunto (\(B\)) más dos espacios vacíos para evitar la adyacencia horizontal entre barcos.
    \end{itemize}

    \item \textbf{Restricciones de las filas:}
    \begin{itemize}
        \item Las filas que contienen subconjuntos tienen una restricción igual a \(B\) (\(r_i = B\)).
        \item Las filas vacías (entre subconjuntos) tienen una restricción igual a 0 (\(r_i = 0\)).
    \end{itemize}

    \item \textbf{Restricciones de las columnas:}
    Las restricciones de las columnas (\(c_j\)) serán los elementos del conjunto \(A = \{a_1, a_2, \dots, a_{3m}\}\). Es decir, cada columna \(j\) tendrá una restricción igual a \(a_j\).

    \item \textbf{Barcos:}
    Introducimos \(3m\) barcos, uno por cada elemento de \(A\), donde el barco \(b_j\) tiene una longitud igual a \(a_j\).
\end{itemize}

\paragraph{Ejemplo de construcción:}

Dado el conjunto \(A = \{5, 2, 7, 5, 2, 1, 9, 1, 1\}\), con \(m = 3\) subconjuntos y \(B = 11\), consideremos los subconjuntos:

\[
S_1 = \{5, 5, 1\}, \quad S_2 = \{7, 2, 2\}, \quad S_3 = \{9, 1, 1\}.
\]

El tablero construido tendrá:

\begin{itemize}
    \item Cantidad de filas = \(m \times 2 - 1 = 5\). Tres para cada uno de los subconjuntos, y dos vacías intercaladas para evitar adyacencias.
    \item Cantidad de columnas = \(B + 2 = 11 + 2 = 13\).
\end{itemize}

El tablero sería:

\[
\begin{array}{ccccccccccccc}
0 & 0 & 0 & 0 & 0 & - & 3 & 3 & 3 & 3 & 3 & - & 5 \\
- & - & - & - & - & - & - & - & - & - & - & - & - \\
2 & 2 & 2 & 2 & 2 & 2 & 2 & - & 1 & 1 & - & 4 & 4 \\
- & - & - & - & - & - & - & - & - & - & - & - & - \\
6 & 6 & 6 & 6 & 6 & 6 & 6 & 6 & 6 & - & 7 & - & 8 \\
\end{array}
\]

\paragraph{Relación con 3-Partition:}
\begin{itemize}
    \item Cada subconjunto \(S_i\) corresponde a una fila del tablero.
    \item Los elementos del subconjunto \(S_i\) corresponden a barcos que ocupan exactamente \(B\) celdas en esa fila.
    \item Si se puede configurar el tablero cumpliendo todas las restricciones, entonces existe una partición válida para \(3\)-Partition, y viceversa.
\end{itemize}

\subsection*{Equivalencia de las instancias}

\begin{itemize}
    \item Si existe una solución para el problema \textit{3-Partition}, entonces podemos construir una configuración válida del tablero para \textit{La Batalla Naval}.
    \item Si existe una configuración válida para \textit{La Batalla Naval}, entonces podemos construir una partición válida para el problema \textit{3-Partition}.
\end{itemize}

\subsection*{Conclusión}

Hemos demostrado que:
\begin{itemize}
    \item \textit{La Batalla Naval} pertenece a NP.
    \item Reducimos un problema NP-completo (\textit{3-Partition}) a \textit{La Batalla Naval} en tiempo polinomial.
\end{itemize}

Por lo tanto, \textit{La Batalla Naval} es NP-completo.
