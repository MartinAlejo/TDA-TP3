\section{Demostración: NP-Completo}


\subsection*{Reducción desde el problema 3-Partition en su version unaria}

Vamos a demostrar que la batalla naval es $NP$-Completo. Pero, ¿Como se demuestra que un problema es $NP$-Completo? Tiene que cumplir dos condiciones: 

\begin{itemize}
    \item \textbf{Pertenencia a NP:} La verificación de una solución candidata es posible en tiempo polinomial.
    \item \textbf{NP-dificultad:} Se puede realizar una reducción polinomial desde cualquier problema NP-completo hacia este problema.
\end{itemize}

Ya en la sección anterior pudimos verificar con éxito que nuestro problema es un problema NP, ahora hay que demostrar que se puede realizar una reducción polinomial desde cualquier problema NP-Completo. Recordemos que todos los NP completos pueden ser reducidos despues cualquier problema NP completo. 

En nuestro caso, utilizaremos el problema de \textit{3-Partition}, que como demostraremos más abajo, es un problema NP-Completo, para verificar que el problema de la batalla naval pertenece a NP-Completo. Es decir, se puede realizar la reduccion polinomial: \textbf{3-Partition} $\leq_p$ \textbf{PBN}.

Definamos un poco cómo se compone el problema \textit{3-Partition}.

\textbf{Nota}: Por simplificacion, siempre que hablemos de 3-Partition, nos estamos refiriendo a la version unaria del problema.

\subsubsection*{Definición del problema 3-Partition}

Dado un conjunto de enteros positivos \(S = \{a_1, a_2, \dots, a_{n}\}\), donde cada número está representado en notación unaria, decide si el conjunto puede partirse en 3 subconjuntos disjuntos \(S_1, S_2, S_3\), tal que la suma de cada subconjunto sea la misma. Es decir:

\begin{itemize}
    \item La suma de los elementos de cada subconjunto \(S_i\) = m
    \item sum(S) = \(sum(S_1) + sum(S_2) + sum(S_3)\) = 3m
\end{itemize}

Veamos un ejemplo: 

$S = \{1, 111, 11, 1, 1, 1\}$

\textbf{Nota}: Es la representacion unaria de $S = \{1, 3, 2, 1, 1, 1\}$

Una solución a esta instancia del problema de 3-Partition es la siguiente:

$S_1 = \{111\}$, $S_2 = \{11, 1\}$, $S_3 = \{1, 1, 1\}$

Vemos que es solución, ya que se cumple que la suma de los elementos en cada subconjunto $S_i = 3$. %Y los subconjuntos S_i son disjuntos.%

\section*{Reducción de 2-Partition a 3-Partition}

\subsection*{Definiciones}

\textbf{2-Partition:} El problema consiste en dividir un conjunto de números \( A = \{a_1, a_2, \dots, a_n\} \) en dos subconjuntos disjuntos \( S_1 \) y \( S_2 \) tales que la suma de los elementos de ambos subconjuntos sea igual.

\textbf{3-Partition:} El problema consiste en dividir un conjunto de números (en representacion unaria) \( B = \{b_1, b_2, \dots, b_n\} \) en tres subconjuntos disjuntos \( S_1, S_2, S_3 \), tales que la suma de los elementos en cada subconjunto sea la misma.

\subsection*{Pertenencia a NP}

Primero hay que demostrar que 3-Partition pertenece a NP, para ello, debemos poder encontrar un validador que valide una solución al problema de 3-Partition en tiempo polinomial. Dicho validador es el siguiente:

Dado \( S = \{a_1, a_2, \dots, a_n\} \), y tres subconjuntos $S_1$, $S_2$, y $S_3$, se valida que:

\begin{itemize}
    \item $S$, $S_1$, $S_2$, $S_3$ estan en representacion unaria (lo estan si su representacion consiste en $a_i$ veces el numero 1)
    \item $sum(S_1)$ = $sum(S_2)$ = $sum(S_3)$
    \item $S_1$ $\cup$ $S_2$ $\cup$ $S_3$ = $S$
    \item $S_1$ $\cap$ $S_2$ $\cap$ $S_3$ = $\emptyset$
\end{itemize}

Dado que verificar la representacion unaria, comprobar que los subconjuntos son disjuntos, calcular la suma de los elementos de cada subconjunto, verificar que la sumas de cada subconjunto son iguales, se puede hacer en tiempo polinomial, decimos que 3-Partition esta en NP.

\subsection*{Reducción}

Para reducir polinomialmente una instancia de \textbf{2-Partition} a una instancia de \textbf{3-Partition}, hacemos lo siguiente:

Consideramos el conjunto \( S = \{a_1, a_2, \dots, a_n\} \) tal que $sum(S) = 2m$. Donde $m = sum(S_1) = sum(S_2)$.

Lo que debemos hacer, es calcular m, y agregarlo al conjunto S. Luego, realizamos la representacion unaria de los elementos de S, y resolvemos el problema usando la caja negra de 3-Partition. Si hay solución para 3-Partion, hay solución para 2-Partition.

Veamoslo con un ejemplo:

$S = \{3, 4, 3, 4\}$. Una solución al problema de 2-Partition es la siguiente: $S_1 = \{3, 4\}$, $S_2 = \{3, 4\}$. Pero vamos a resolverlo usando 3-Partition de la forma que mencionamos arriba, es decir, vamos a transformar esta instancia a una que pueda resolverse con 3-Partition.

Vemos que $m = sum(S)$ $/$ $2 = (3 + 4 + 3 + 4)$ $/$ $2 = 7$, entonces al agregar a $m$ al conjunto $S$ y representar todo de forma unaria, tenemos $S = \{111, 1111, 111, 1111, 1111111\}$. Y usando la caja negra de 3-Partition, vemos que hay solución. De hecho, es la siguiente:

$S_1 = \{111, 1111\}$, $S_2 = \{111, 1111\}$, $S_3 = \{1111111\}$.

\textbf{Nota}: Siempre vamos a encontrar que el tercer subconjunto va a ser exactamente m, y por lo tanto, sumar exactamente m tambien (que es lo que por supuesto, suman los otros 2 subconjuntos).

Si pudimos encontrar estos 3 subconjuntos, implica que hay solución para 2-Partition, ya que ahora en 3-Partition $sum(S) = 3m = 2m + m$, y esto solo sucede si $sum(S_1) = sum(S_2) = sum(S_3) = m$, que significa que en la instancia original de 2-Partition existe solución tal que $sum(S) = 2m$, siendo $sum(S_1) = m$, $sum(S_2) = m$.

Dado que encontrar m, agregar m al conjunto S y representar a los elementos de S en forma unaria, pueden hacerse en tiempo polinomial. Entonces pudimos hacer la reduccion \textbf{2-Partition} $\leq_p$ \textbf{3-Partition}. Quedando demostrado que 3-Partition pertenece a NP-Completo.

% DE ACA PARA ABAJO LA MAYOR PARTE ESTA MAL (HAY QUE REHACER USANDO LA VERSION UNARIA DE LA VARIACION DEL 3-PARTITION)

\subsection*{Reducción de 3-Partition a La Batalla Naval}

Dado una instancia del problema \textit{3-Partition}, construiremos una instancia del problema \textit{La Batalla Naval}.

\paragraph{Construcción del tablero:}
\begin{itemize}
    \item \textbf{Dimensiones del tablero:} Construimos un tablero con \(m\) filas (una fila para cada subconjunto \(S_i\)) y \(3m\) columnas (una columna para cada elemento de A).
    \item \textbf{Restricciones de las filas:} Las restricciones de las filas (\(r_i\)) serán \(B\), es decir, cada fila \(i\) debe contener exactamente \(B\) casillas ocupadas.
    \item \textbf{Restricciones de las columnas:} Las restricciones de las columnas (\(c_j\)) serán los elementos del conjunto \(A = \{a_1, a_2, \dots, a_{3m}\}\). Es decir, la columna \(j\) tendrá una restricción igual a \(a_j\).
    \item \textbf{Barcos:} Introducimos \(3m\) barcos, uno por cada elemento de \(A\), donde el barco \(b_j\) tiene una longitud igual a \(a_j\).
\end{itemize}

\paragraph{Ejemplo de construcción:}
Veamos con un ejemplo para poner en contexto lo que estamos diciendo:

\vskip0.5cm
Tenemos los siguientes datos del problema 3-Partition: 

\begin{itemize}
    \item $A = \{6, 5, 4, 5, 3, 7\}$,
    \item $B = 15$, porque por ejemplo si tengo los subconjuntos $S_{1}={6,5,4}$ y $S_{2}={5,3,7}$ la suma de los elementos de $S_{1}$ es 15 al igual que la suma de los elementos de $S_{2}$ 
    \item $m = 2$, porque $|A| = 6 = 3 \times 2 = 3 \times m$
\end{itemize}

\subsection*{Ajustando el Tablero}

Para que cada fila pueda contener exactamente $r_i = 15$ casillas ocupadas, necesitamos que el número de columnas del tablero sea suficiente para permitir esa suma. Ajustamos las dimensiones del tablero:

\begin{itemize}
    \item $n = 2$ filas (porque $m = 2$, el número de subconjuntos). % Quiza habria que agregar filas vacias intercaladas
    \item $m = 15$ columnas (ya que cada fila debe contener hasta 15 casillas ocupadas). % Quiza habria que agregar siempre 2 columnas
\end{itemize}

Por lo tanto, el tablero tiene $2 \times 15 = 30$ casillas, lo cual coincide con la suma total de $\sum A = 30$. % Entiuendo que estaria mal tras hacer las correcciones de arriba %

\subsection*{Nueva Configuración}

\begin{enumerate}
    \item \textbf{Restricciones de Filas y Columnas:}
    \begin{itemize}
        \item Cada fila $i$ debe tener exactamente $r_i = B = 15$ casillas ocupadas.
        \item Cada columna $j$ debe contener exactamente $c_j = a_j$, con $A = \{6, 5, 4, 5, 3, 7\}$.
    \end{itemize}

    \item \textbf{Barcos:}
    Introducimos un barco para cada elemento en $A$:
    \begin{itemize}
        \item Barco 1: longitud 6,
        \item Barco 2: longitud 5,
        \item Barco 3: longitud 4,
        \item Barco 4: longitud 5,
        \item Barco 5: longitud 3,
        \item Barco 6: longitud 7.
    \end{itemize}

    \item \textbf{Restricciones Adicionales:}
    \begin{itemize}
        \item Los barcos no pueden ser adyacentes entre sí (ni horizontal, ni vertical, ni diagonalmente).
        \item Todos los barcos deben estar dentro del tablero, y la suma total de las celdas ocupadas debe ser $\sum A = 30$.
    \end{itemize}
\end{enumerate}

\subsection*{Solución del Problema de Batalla Naval}

Dado el tablero con $n = 2$ filas y $m = 15$ columnas, encontramos la siguiente distribución:

\begin{itemize}
    \item \textbf{Fila 1 ($S_1$):} Barcos $\{6, 5, 4\}$, que suman $6 + 5 + 4 = 15$.
    \item \textbf{Fila 2 ($S_2$):} Barcos $\{5, 3, 7\}$, que suman $5 + 3 + 7 = 15$.
\end{itemize}

Esto satisface las restricciones de las filas ($r_i = 15$) y las columnas ($c_j = a_j$).

\subsection*{Relación entre 3-Partition y Batalla Naval}

\begin{enumerate}
    \item Resolver el problema de \textit{La Batalla Naval} (encontrar la disposición de los barcos en el tablero) equivale a resolver el problema de 3-Partition:
    \begin{itemize}
        \item Cada fila representa un subconjunto de $A$.
        \item La longitud de los barcos en cada fila corresponde a los elementos del subconjunto.
        \item La suma de las longitudes de los barcos en cada fila debe ser igual a $B = 15$.
    \end{itemize}

    \item Si no existe una solución para el tablero naval, no existe una partición válida en el problema de 3-Partition.
\end{enumerate}


\subsubsection*{Equivalencia de las instancias}

\begin{itemize}
    \item Si existe una solución para el problema \textit{3-Partition}, entonces podemos construir una configuración válida del tablero para \textit{La Batalla Naval}.
    \item Si existe una configuración válida para \textit{La Batalla Naval}, entonces podemos construir una partición válida para el problema \textit{3-Partition}.
\end{itemize}

\subsection*{Conclusión}

Hemos demostrado que:
\begin{itemize}
    \item \textit{La Batalla Naval} pertenece a NP.
    \item Reducimos un problema NP-completo (\textit{3-Partition}) a \textit{La Batalla Naval} en tiempo polinomial.
\end{itemize}

Por lo tanto, \textit{La Batalla Naval} es NP-completo.