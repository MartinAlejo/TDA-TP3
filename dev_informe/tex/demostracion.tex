\section{Demostración: NP-Completo}

Vamos a demostrar que la batalla naval es $NP$ completo. Pero, ¿Como se demuestra que un algortimo es $NP$ completo? Tiene que cumplir dos condiciones: 

\begin{itemize}
    \item \textbf{Pertenencia a NP:} La verificación de una solución candidata es posible en tiempo polinomial.
    \item \textbf{NP-dificultad:} Se puede realizar una reducción polinomial desde cualquier problema NP-completo hacia este problema.
\end{itemize}

En la sección anterior pudimos verificar con éxito que nuestro problema es de tipo NP, ahora hay que demostrar que se puede realizar una reducción polinomial desde cualquier problema NP-Completo. Recordemos que todos los problemas NP-Completos pueden reducirse a cualquier problema NP-Completo. 

En nuestro caso, utilizaremos el problema de \textit{2-Partition}, que como fue demostrado anteriormente en clase, es un problema NP-Completo, para verificar que el problema de la batalla naval pertenece a NP-Completo. Es decir, se puede realizar la reduccion polinomial: \textbf{2-Partition} $\leq_p$ \textbf{PBN}
\vskip0.5cm
\subsection*{Definiciones}

\textbf{2-Partition}: Se tiene un conjunto $S$ de enteros positivos, y dos subconjuntos $S_1$ y $S_2$, tales que se cumple:
\begin{itemize}
    \item Los subconjuntos son disjuntos
    \item La unión de los subconjuntos es el conjunto original
    \item $\text{sum}(S_1) + \text{sum}(S_2) = \text{sum}(S)$
    \item $\text{sum}(S_1) = \text{sum}(S_2)$
\end{itemize}

\textbf{Ejemplo}:

$S =\{3, 3, 4, 4\},\quad S_1 = \{3, 4\},\quad S_2 = \{3, 4\} \quad \rightarrow \quad \text{sum}(S)= 14 = \text{sum}(S_1) + \text{sum}(S_2) = 7 + 7$

\vskip0.5cm
\textbf{Problema de la Batalla Naval}: Dado un tablero de $n \times m$, una lista de $k$ barcos (donde el barco $i$ tiene longitud $b_i$), una lista de restricciones para filas (demandas de casilleros ocupados en cada fila) y otra para columnas, determinar si existe una forma válida de ubicar los barcos en el tablero cumpliendo las restricciones, sin que los barcos sean adyacentes.

\subsection*{Planteo del problema}

Vamos a utilizar el conjunto visto anteriormente ($S = \{3,3,4,4\}$), o sea una instancia del problema de 2-Partition, la cual resolveremos utilizando el problema de la batalla naval.

Para ello, debemos crear un tablero de dimensiones $i \times j$, donde $i$ son la cantidad de filas, y $j$ la cantidad de columnas.

\textbf{Dimensiones del tablero:}
\begin{itemize}
    \item Cantidad de filas ($i$) = $sum(S) + n - 1  = (T + 4 - 1 ) = 17$
    \item Cantidad de columnas ($j$) = $2$ 
\end{itemize}
\vskip0.5cm

\textbf{Nota}: Siendo $n$ la cantidad de elementos del conjunto $S$, y $T = \sum_{i=1}^{n} a_i \quad \text{con} \quad a_i \in S$

\vskip0.25cm

\textbf{Nota}: La cantidad de columnas proviene de la cantidad de subconjuntos (2).

\vskip0.5cm

\textbf{Modelo del problema}:

\begin{itemize}
    \item \textbf{Lista de barcos:} Cada elemento $a_i \in S$ corresponde a un barco de longitud $a_i$. Para nuestro ejemplo, los barcos serían de longitudes $3, 3, 4, 4$.

    \item \textbf{Restricciones de las columnas:}
    Cada columna debe contener exactamente $T / 2$ casilleros ocupados. 

    \item \textbf{Restricciones de las filas:}
    Cada fila puede contener como máximo un casillero ocupado. Al tener $n - 1$ filas adicionales, nos aseguramos que los barcos no sean adyacentes entre sí. Ya que tras colocar un barco dejamos una fila vacía.
\end{itemize}

\vskip0.25cm

Ejemplo del tablero: 

\begin{center}
\[
    \begin{bmatrix}
    1 & \text{-} \\
    1 & \text{-} \\
    1 & \text{-} \\
    \text{-} & \text{-} \\
    \text{-} & 2 \\
    \text{-} & 2 \\
    \text{-} & 2 \\
    \text{-} & \text{-} \\
    3 & \text{-} \\
    3 & \text{-} \\
    3 & \text{-} \\
    3 & \text{-} \\
    \text{-} & \text{-} \\
    \text{-} & 4 \\
    \text{-} & 4 \\
    \text{-} & 4 \\
    \text{-} & 4
    \end{bmatrix}
\]
\end{center}

Los barcos pertenecientes a la primer columna = $S_{1} = {a_{1}, a_{2},...,a_{q}} = \{3, 4\}$ 
\vskip0.25cm
Los barcos pertenecientes a la segunda columna = $S_{2} = {a_{q +1}, a_{q + 2},..., a_{n}} = \{3, 4\}$
\vskip0.5cm
\subsection*{Demostración de la Reducción}

\paragraph{ Si existe una solución para 2-Partition:}
  
Sea $S_1$ y $S_2$ la partición de $S$ tal que $\text{sum}(S_1) = \text{sum}(S_2) = T / 2$.  

En el problema de la Batalla Naval:

\begin{itemize}
    \item Colocamos los barcos pertenecientes a los elementos de $S_1$ en la primera columna.
    \item Colocamos los barcos pertenecientes a los elementos de $S_2$ en la segunda columna.
    \item Aseguramos que entre dos barcos haya al menos una fila vacía, cumpliendo la condición de no adyacencia.
\end{itemize}

Como cada columna tiene exactamente $T / 2$ casilleros ocupados y los barcos están distribuidos sin ser adyacentes, la instancia del problema de la Batalla Naval tiene una solución válida.

\paragraph{Si existe una solución para el problema de la Batalla Naval:}


Sea una distribución válida de los barcos en el tablero.  

\begin{itemize}
    \item Los barcos en la primera columna representan un subconjunto $S_1$ de $S$, y los barcos en la segunda columna representan otro subconjunto $S_2$.
    \item Como cada columna tiene exactamente $T / 2$ casilleros ocupados, tenemos que $\text{sum}(S_1) = \text{sum}(S_2) = T / 2$.
    \item $S_1$ y $S_2$ al ser disjuntos, la unión de estos da como resultado el conjunto $S$, resolviendo la instancia de 2-Partition.
\end{itemize}

\subsection*{Conclusión}

La reducción transforma una instancia de 2-Partition en una instancia del problema de la Batalla Naval en tiempo polinomial, ya que la construcción del tablero y las restricciones toma tiempo proporcional a $n$, el tamaño del conjunto $S$.

Por lo tanto, hemos desmotrado que es posible realizar la reducción polinomial: 2-Partition $\leq_p$ PBN.
