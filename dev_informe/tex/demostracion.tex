\section{Demostración: NP-Completo}


\subsection*{Reducción desde el problema 2-Partition}

Vamos a demostrar que la batalla naval es $NP$ completo. Pero, ¿Como se demuestra que un algortimo es $NP$ completo? Tiene que cumplir dos puntos: 

\begin{itemize}
    \item \textbf{Pertenencia a NP:} La verificación de una solución candidata es posible en tiempo polinomial.
    \item \textbf{NP-dificultad:} Se puede realizar una reducción polinomial desde cualquier problema NP-completo hacia este problema.
\end{itemize}

Ya en la sección anterior pudimos verificar con éxito que nuestro problema es de tipo NP, ahora hay que demostrar que se puede realizar una reducción polinomial desde cualquier problema NP-Completo. Recordemos que todos los NP completos pueden ser reducidos despues a cualquier problema NP completo. 

En nuestro caso, utilizaremos el problema de \textit{2-Partition}, que como fue demostrado anteriormente en clase, es un problema NP-Completo, para verificar que el problema de la batalla naval puede ser resuelto en tiempo polinomial. 


Definamos un poco cómo se compone el problema \textit{2-Partition}


Se tiene un conjunto $S$ y dos subconjuntos $S_1$ y $S_2$, tales que se tengan que cumplir algunas propiedades:
\begin{itemize}
    \item $\text{sum}(S_1) + \text{sum}(S_2) = \text{sum}(S)$. siendo $\text{sum}(S_1) = \text{sum}(S_2)$
    \item Los subconjuntos son disjuntos
    \item La unión de los subconjuntos es el conjunto original
\end{itemize}

\textbf{Ejemplo}:
$S = \{4, 2, 9, 6, 1, 8\}, \quad S_1 = \{4, 2, 9\}, \quad S_2 = \{6, 1, 8\} \quad \rightarrow \quad \text{sum}(S) = 30 = \text{sum}(S_1) + \text{sum}(S_2) = 15 + 15.$

Vamos a realizar la reduccion polinomial: $\text{2\-Partition} \leq_p \text{PBN}$

