\section{Demostración: NP-Completo}


\subsection*{Reducción desde el problema 3-Partition}

Vamos a demostrar que la batalla naval es NP completo. Pero, ¿Como se demuestra que un algortimo es NP completo? Tiene que cumplir dos puntos: 

\begin{itemize}
    \item \textbf{Pertenencia a NP:} La verificación de una solución candidata es posible en tiempo polinomial.
    \item \textbf{NP-dificultad:} Se puede realizar una reducción polinomial desde cualquier problema NP-completo hacia este problema.
\end{itemize}

Ya en la sección anterior pudimos verificar con éxito que nuestro problema es un problema NP, ahora hay que demostrar que se puede realizar una reducción polinomial desde cualquier problema NP-Completo. Recordemos que todos los NP completos pueden ser reducidos despues cualquier problema NP completo. 

En nuestro caso, utilizamos el problema de \textit{3-Partition} para verificar que la batalla naval puede ser resuelta en tiempo logarítmico.


Definamos un poco cómo se compone el problema \textit{3-Partition}



\subsubsection*{Definición del problema 3-Partition}

Dado un conjunto de \(3m\) números positivos \(A = \{a_1, a_2, \dots, a_{3m}\}\), donde cada número está representado en notación unaria y la suma total de los números es \(S = m \cdot B\), queremos particionar \(A\) en \(m\) subconjuntos disjuntos \(S_1, S_2, \dots, S_m\) tales que:
\begin{itemize}
    \item Cada subconjunto \(S_i\) tiene exactamente 3 elementos.
    \item La suma de los elementos en cada subconjunto es exactamente \(B\).
\end{itemize}

\subsubsection*{Reducción de 3-Partition a La Batalla Naval}

Dado una instancia del problema \textit{3-Partition}, construiremos una instancia del problema \textit{La Batalla Naval}.

\paragraph{Construcción del tablero:}
\begin{itemize}
    \item \textbf{Dimensiones del tablero:} Construimos un tablero con \(m\) filas y \(3m\) columnas. Las restricciones de las filas (\(r_i\)) serán \(B\), es decir, cada fila \(i\) debe contener exactamente \(B\) casillas ocupadas.
    \item \textbf{Restricciones de las columnas:} Las restricciones de las columnas (\(c_j\)) serán los elementos del conjunto \(A = \{a_1, a_2, \dots, a_{3m}\}\). Es decir, la columna \(j\) tendrá una restricción igual a \(a_j\).
    \item \textbf{Barcos:} Introducimos \(3m\) barcos, uno por cada elemento de \(A\), donde el barco \(b_j\) tiene una longitud igual a \(a_j\).
\end{itemize}

vamos con un ejemplo para poner en contexto esto que estamos diciendo:

\vskip0.5cm
Tenemos los siguientes datos del problema 3-Partition: 

\begin{itemize}
    \item $A = \{6, 5, 4, 5, 3, 7\}$,
    \item $B = 15$, porque por ejemplo si tengo los subconjuntos $S_{1}={6,5,4}$ y $S_{2}={5,3,7}$ la suma de los elementos de $S_{1}$ es 15 al igual que la suma de los elementos de $S_{2}$ 
    \item $m = 2$. porque $|A| = 6 = 3.2 => 3.m$
\end{itemize}

\subsection*{Ajustando el Tablero}

Para que cada fila pueda contener exactamente $r_i = 15$ casillas ocupadas, necesitamos que el número de columnas del tablero sea suficiente para permitir esa suma. Ajustamos las dimensiones del tablero:

\begin{itemize}
    \item $n = 2$ filas (porque $m = 2$, el número de subconjuntos).
    \item $m = 15$ columnas (ya que cada fila debe contener hasta 15 casillas ocupadas).
\end{itemize}

Por lo tanto, el tablero tiene $2 \times 15 = 30$ casillas, lo cual coincide con la suma total de $\sum A = 30$.

\subsection*{Nueva Configuración}

\begin{enumerate}
    \item \textbf{Restricciones de Filas y Columnas:}
    \begin{itemize}
        \item Cada fila $i$ debe tener exactamente $r_i = B = 15$ casillas ocupadas.
        \item Cada columna $j$ debe contener exactamente $c_j = a_j$, con $A = \{6, 5, 4, 5, 3, 7\}$.
    \end{itemize}

    \item \textbf{Barcos:}
    Introducimos un barco para cada elemento en $A$:
    \begin{itemize}
        \item Barco 1: longitud 6,
        \item Barco 2: longitud 5,
        \item Barco 3: longitud 4,
        \item Barco 4: longitud 5,
        \item Barco 5: longitud 3,
        \item Barco 6: longitud 7.
    \end{itemize}

    \item \textbf{Restricciones Adicionales:}
    \begin{itemize}
        \item Los barcos no pueden ser adyacentes entre sí (ni horizontal, ni vertical, ni diagonalmente).
        \item Todos los barcos deben estar dentro del tablero, y la suma total de las celdas ocupadas debe ser $\sum A = 30$.
    \end{itemize}
\end{enumerate}

\subsection*{Solución del Problema de Batalla Naval}

Dado el tablero con $n = 2$ filas y $m = 15$ columnas, encontramos la siguiente distribución:

\begin{itemize}
    \item \textbf{Fila 1 ($S_1$):} Barcos $\{6, 5, 4\}$, que suman $6 + 5 + 4 = 15$.
    \item \textbf{Fila 2 ($S_2$):} Barcos $\{5, 3, 7\}$, que suman $5 + 3 + 7 = 15$.
\end{itemize}

Esto satisface las restricciones de las filas ($r_i = 15$) y las columnas ($c_j = a_j$).

\subsection*{Relación entre 3-Partition y Batalla Naval}

\begin{enumerate}
    \item Resolver el problema de \textit{La Batalla Naval} (encontrar la disposición de los barcos en el tablero) equivale a resolver el problema de 3-Partition:
    \begin{itemize}
        \item Cada fila representa un subconjunto de $A$.
        \item La longitud de los barcos en cada fila corresponde a los elementos del subconjunto.
        \item La suma de las longitudes de los barcos en cada fila debe ser igual a $B = 15$.
    \end{itemize}

    \item Si no existe una solución para el tablero naval, no existe una partición válida en el problema de 3-Partition.
\end{enumerate}


\subsubsection*{Equivalencia de las instancias}

\begin{itemize}
    \item Si existe una solución para el problema \textit{3-Partition}, entonces podemos construir una configuración válida del tablero para \textit{La Batalla Naval}.
    \item Si existe una configuración válida para \textit{La Batalla Naval}, entonces podemos construir una partición válida para el problema \textit{3-Partition}.
\end{itemize}

\subsection*{Conclusión}

Hemos demostrado que:
\begin{itemize}
    \item \textit{La Batalla Naval} pertenece a NP.
    \item Reducimos un problema NP-completo (\textit{3-Partition}) a \textit{La Batalla Naval} en tiempo polinomial.
\end{itemize}

Por lo tanto, \textit{La Batalla Naval} es NP-completo.
