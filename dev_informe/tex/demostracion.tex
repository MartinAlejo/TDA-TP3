\section{Demostración: NP-Completo}


\subsection*{Reducción desde el problema 3-Partition en su version unaria}

Vamos a demostrar que la batalla naval es $NP$-Completo. Pero, ¿Como se demuestra que un problema es $NP$-Completo? Tiene que cumplir dos condiciones: 

\begin{itemize}
    \item \textbf{Pertenencia a NP:} La verificación de una solución candidata es posible en tiempo polinomial.
    \item \textbf{NP-dificultad:} Se puede realizar una reducción polinomial desde cualquier problema NP-completo hacia este problema.
\end{itemize}

Ya en la sección anterior pudimos verificar con éxito que nuestro problema es un problema NP, ahora hay que demostrar que se puede realizar una reducción polinomial desde cualquier problema NP-Completo. Recordemos que todos los NP completos pueden ser reducidos despues cualquier problema NP completo. 

En nuestro caso, utilizaremos el problema de \textit{3-Partition}, que como demostraremos más abajo, es un problema NP-Completo, para verificar que el problema de la batalla naval pertenece a NP-Completo. Es decir, se puede realizar la reduccion polinomial: \textbf{3-Partition} $\leq_p$ \textbf{PBN}.

Definamos un poco cómo se compone el problema \textit{3-Partition}.

\textbf{Nota}: Por simplificacion, siempre que hablemos de 3-Partition, nos estamos refiriendo a la version unaria del problema.

\subsubsection*{Definición del problema 3-Partition}

Dado un conjunto de enteros positivos \(S = \{a_1, a_2, \dots, a_{n}\}\), donde cada número está representado en notación unaria, decide si el conjunto puede partirse en 3 subconjuntos disjuntos \(S_1, S_2, S_3\), tal que la suma de cada subconjunto sea la misma. Es decir:

\begin{itemize}
    \item La suma de los elementos de cada subconjunto \(S_i\) = m
    \item sum(S) = \(sum(S_1) + sum(S_2) + sum(S_3)\) = 3m
\end{itemize}

Veamos un ejemplo: 

$S = \{1, 111, 11, 1, 1, 1\}$

\textbf{Nota}: Es la representacion unaria de $S = \{1, 3, 2, 1, 1, 1\}$

Una solución a esta instancia del problema de 3-Partition es la siguiente:

$S_1 = \{111\}$, $S_2 = \{11, 1\}$, $S_3 = \{1, 1, 1\}$

Vemos que es solución, ya que se cumple que la suma de los elementos en cada subconjunto $S_i = 3$. %Y los subconjuntos S_i son disjuntos.%

\section*{Reducción de 2-Partition a 3-Partition}

\subsection*{Definiciones}

\textbf{2-Partition:} El problema consiste en dividir un conjunto de números \( A = \{a_1, a_2, \dots, a_n\} \) en dos subconjuntos disjuntos \( S_1 \) y \( S_2 \) tales que la suma de los elementos de ambos subconjuntos sea igual.

\textbf{3-Partition:} El problema consiste en dividir un conjunto de números (en representacion unaria) \( B = \{b_1, b_2, \dots, b_n\} \) en tres subconjuntos disjuntos \( S_1, S_2, S_3 \), tales que la suma de los elementos en cada subconjunto sea la misma.

\subsection*{Pertenencia a NP}

Primero hay que demostrar que 3-Partition pertenece a NP, para ello, debemos poder encontrar un validador que valide una solución al problema de 3-Partition en tiempo polinomial. Dicho validador es el siguiente:

Dado \( S = \{a_1, a_2, \dots, a_n\} \), y tres subconjuntos $S_1$, $S_2$, y $S_3$, se valida que:

\begin{itemize}
    \item $S$, $S_1$, $S_2$, $S_3$ estan en representacion unaria (lo estan si su representacion consiste en $a_i$ veces el numero 1)
    \item $sum(S_1)$ = $sum(S_2)$ = $sum(S_3)$
    \item $S_1$ $\cup$ $S_2$ $\cup$ $S_3$ = $S$
    \item $S_1$ $\cap$ $S_2$ $\cap$ $S_3$ = $\emptyset$
\end{itemize}

Dado que verificar la representacion unaria, comprobar que los subconjuntos son disjuntos, calcular la suma de los elementos de cada subconjunto, verificar que la sumas de cada subconjunto son iguales, se puede hacer en tiempo polinomial, decimos que 3-Partition esta en NP.

\subsection*{Reducción}

Para reducir polinomialmente una instancia de \textbf{2-Partition} a una instancia de \textbf{3-Partition}, hacemos lo siguiente:

Consideramos el conjunto \( S = \{a_1, a_2, \dots, a_n\} \) tal que $sum(S) = 2m$. Donde $m = sum(S_1) = sum(S_2)$.

Lo que debemos hacer, es calcular m, y agregarlo al conjunto S. Luego, realizamos la representacion unaria de los elementos de S, y resolvemos el problema usando la caja negra de 3-Partition. Si hay solución para 3-Partion, hay solución para 2-Partition.

Veamoslo con un ejemplo:

$S = \{3, 4, 3, 4\}$. Una solución al problema de 2-Partition es la siguiente: $S_1 = \{3, 4\}$, $S_2 = \{3, 4\}$. Pero vamos a resolverlo usando 3-Partition de la forma que mencionamos arriba, es decir, vamos a transformar esta instancia a una que pueda resolverse con 3-Partition.

Vemos que $m = sum(S)$ $/$ $2 = (3 + 4 + 3 + 4)$ $/$ $2 = 7$, entonces al agregar a $m$ al conjunto $S$ y representar todo de forma unaria, tenemos $S = \{111, 1111, 111, 1111, 1111111\}$. Y usando la caja negra de 3-Partition, vemos que hay solución. De hecho, es la siguiente:

$S_1 = \{111, 1111\}$, $S_2 = \{111, 1111\}$, $S_3 = \{1111111\}$.

\textbf{Nota}: Siempre vamos a encontrar que un tercer subconjunto va a ser exactamente m, y por lo tanto, sumar exactamente m tambien (que es lo que por supuesto, suman los otros 2 subconjuntos).

Si pudimos encontrar estos 3 subconjuntos, implica que hay solución para 2-Partition, ya que ahora en 3-Partition $sum(S) = 3m = 2m + m$, y esto solo sucede si $sum(S_1) = sum(S_2) = sum(S_3) = m$, que significa que en la instancia original de 2-Partition existe solución tal que $sum(S) = 2m$, siendo $sum(S_1) = m$, $sum(S_2) = m$.

Dado que encontrar m, agregar m al conjunto S y representar a los elementos de S en forma unaria, pueden hacerse en tiempo polinomial. Entonces pudimos hacer la reduccion \textbf{2-Partition} $\leq_p$ \textbf{3-Partition}. Quedando demostrado que 3-Partition pertenece a NP-Completo.

% DE ACA PARA ABAJO LA MAYOR PARTE ESTA MAL (HAY QUE REHACER VARIAS COSAS USANDO LA VERSION UNARIA DE LA VARIACION DEL 3-PARTITION)

\subsection*{Reducción de 3-Partition a La Batalla Naval}

Dado una instancia del problema \textit{3-Partition}, construiremos una instancia del problema \textit{La Batalla Naval}.

\paragraph{Construcción del tablero:}

\begin{itemize}
    \item \textbf{Dimensiones del tablero:} % IMPORTANTE: Quiza invirtiendo la logica entre filas/columnas sea mas facil o igual %
    \begin{itemize}
        \item Cantidad de filas: 5. Cada subconjunto \(S_i\) ocupa una fila, y se agrega una fila vacía después de cada subconjunto (excepto después del último).
        \item Cantidad de columnas: $sum(S)$ = \(3m\). % NO ESTOY SEGURO %
    \end{itemize}

    \item \textbf{Restricciones de las filas:}
    \begin{itemize}
        \item Las filas que contienen subconjuntos tienen una restricción igual a $m$.
        \item Las filas vacías (entre subconjuntos) tienen una restricción igual a 0 (\(r_i = 0\)).
    \end{itemize}

    \item \textbf{Restricciones de las columnas:} Las restricciones de las columnas van a ser 1 para todas. Ya que en cada columna va a haber exactamente una parte de un barco.

    \item \textbf{Barcos:} Los barcos van a ser los elementos $a_i$ que pertenecen al conjunto S.
\end{itemize}

% ME QUEDE ACA

\paragraph{Ejemplo de construcción:}

Dado el conjunto \(A = \{5, 2, 7, 5, 2, 1, 9, 1, 1\}\), con \(m = 3\) subconjuntos y \(B = 11\), consideremos los subconjuntos:

\[
S_1 = \{5, 5, 1\}, \quad S_2 = \{7, 2, 2\}, \quad S_3 = \{9, 1, 1\}.
\]

El tablero construido tendrá:

\begin{itemize}
    \item Cantidad de filas = \(m \times 2 - 1 = 5\). Tres para cada uno de los subconjuntos, y dos vacías intercaladas para evitar adyacencias.
    \item Cantidad de columnas = \(B + 2 = 11 + 2 = 13\).
\end{itemize}

El tablero sería:

\[
\begin{array}{ccccccccccccc}
0 & 0 & 0 & 0 & 0 & - & 3 & 3 & 3 & 3 & 3 & - & 5 \\
- & - & - & - & - & - & - & - & - & - & - & - & - \\
2 & 2 & 2 & 2 & 2 & 2 & 2 & - & 1 & 1 & - & 4 & 4 \\
- & - & - & - & - & - & - & - & - & - & - & - & - \\
6 & 6 & 6 & 6 & 6 & 6 & 6 & 6 & 6 & - & 7 & - & 8 \\
\end{array}
\]

\paragraph{Relación con 3-Partition:}
\begin{itemize}
    \item Cada subconjunto \(S_i\) corresponde a una fila del tablero.
    \item Los elementos del subconjunto \(S_i\) corresponden a barcos que ocupan exactamente \(B\) celdas en esa fila.
    \item Si se puede configurar el tablero cumpliendo todas las restricciones, entonces existe una partición válida para \(3\)-Partition, y viceversa.
\end{itemize}

\subsection*{Equivalencia de las instancias}

\begin{itemize}
    \item Si existe una solución para el problema \textit{3-Partition}, entonces podemos construir una configuración válida del tablero para \textit{La Batalla Naval}.
    \item Si existe una configuración válida para \textit{La Batalla Naval}, entonces podemos construir una partición válida para el problema \textit{3-Partition}.
\end{itemize}

\subsection*{Conclusión}

Hemos demostrado que:
\begin{itemize}
    \item \textit{La Batalla Naval} pertenece a NP.
    \item Reducimos un problema NP-completo (\textit{3-Partition}) a \textit{La Batalla Naval} en tiempo polinomial.
\end{itemize}

Por lo tanto, \textit{La Batalla Naval} es NP-completo.